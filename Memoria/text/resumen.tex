Las reglas de asociación son objetos matemáticos empleados de forma extensa en disciplinas como la minería de datos, aprendizaje automático y representación del conocimiento, entre otros campos.
Slatt es un proyecto de software libre desarrollado por José Luis Balcázar (Universidad Politécnica de Barcelona). Ofrece funcionalidades para el cálculo de reglas de asociación. Para ello, se apoya en implementaciones del algoritmo a priori para el cálculo de clausuras, el retículo de las clausuras y, entre otras
funcionalidades,devuelve las reglas representativas para cualquier elección de los parámetros de soporte y confianza.

En este proyecto, se ha mejorado este software utilizando como base las implementaciones disponibles en Slatt aplicadas a:

\begin{itemize}
    \item hipergrafos y algoritmos con aplicación a estos objetos;
    
    \item cálculo de clausuras y retículos (lattices);
\end{itemize}


Este trabajo ha requerido la búsqueda y análisis de algoritmos propuestos en la literatura científica sobre los puntos anteriores.
El lenguaje de desarrollo será Python3, encontrándose la implementación anterior en Python 2.7
%
% Conviene evitar aquí las llamadas a la bibliografía del
% trabajo, ya que el resumen tiene entidad independiente.
%

%% Aportamos nuestra perspectiva sobre la pertinencia de las
%% instrucciones {\tt goto}.


%%Pasar a tiempo presente
